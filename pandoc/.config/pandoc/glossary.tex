\newglossaryentry{discrepantie}
{   name={discrepantie},
    description={Een situatie waarin 2 zaken niet met elkaar overeenstemmen}
}
\newglossaryentry{fitnessen}
{   name={fitnessen},
    description={Fitness is een geheel van sporten en oefeningen die thuis of in de sportschool worden uitgeoefend. Een fitness training kan er op gericht zijn om bepaalde spieren of spiergroepen te trainen, of juist het uithoudingsvermogen en de conditie te verbeteren}
}
\newglossaryentry{apparatuur}
{   name={apparatuur},
    description={Geheel van bij elkaar horende apparaten (bv. yoga mat)}
}
\newglossaryentry{headless}
{   name={headless CMS},
    description={A headless CMS is a back-end only content management system (CMS) built from the ground up as a content repository that makes content accessible via a RESTful API for display on any device.}
}
\newglossaryentry{lifting belt}
{   name={lifting belt},
    description={Een dikke riem die onderstuining biedt aan de kern van het lichaam, dit zorgt voor een stabielere positie van de rug.}
}
\newglossaryentry{lifting straps}
{   name={lifting straps},
    description={Een lap stof die je om een stang wikkelt voor het bevorderen van de grip.}
}
\newglossaryentry{relationele database}
{   name={relationele database},
    description={Een relationele database is een database die is opgebouwd volgens een relationeel model. De gegevens worden opgeslagen in tabellen waarin de rijen de soortgelijke groepen informatie, de records vormen, en de kolommen de informatie die voor elk record moet worden opgeslagen.}
}
\newglossaryentry{react}
{   name={react},
    description={React is een Javascriptbibliotheek om gebruikersinterfaces mee te bouwen. React werd in 2011 door Facebook ontwikkeld vanuit de behoefte om de code van grote webapplicaties beter beheersbaar te maken.}
}
\newacronym{CMS}{CMS}{Content management system}
